\documentclass[12pt,a4paper]{article}
\usepackage{ctex}  % 支持中文
\usepackage{graphicx}  % 插入图片
\usepackage{amsmath,amssymb}  % 数学公式
\usepackage{booktabs}  % 表格
\usepackage{hyperref}  % 超链接
\usepackage{geometry}  % 页面设置
\usepackage{listings}  % 代码
\usepackage{color}  % 颜色
\usepackage{xcolor}  % 增强颜色
\usepackage{subcaption}  % 子图
\usepackage{float}  % 控制浮动体

% 页面设置
\geometry{left=2.5cm,right=2.5cm,top=2.5cm,bottom=2.5cm}

% 代码环境设置
\definecolor{codegreen}{rgb}{0,0.6,0}
\definecolor{codegray}{rgb}{0.5,0.5,0.5}
\definecolor{codepurple}{rgb}{0.58,0,0.82}
\definecolor{backcolour}{rgb}{0.95,0.95,0.95}

\lstdefinestyle{mystyle}{
    backgroundcolor=\color{backcolour},   
    commentstyle=\color{codegreen},
    keywordstyle=\color{magenta},
    numberstyle=\tiny\color{codegray},
    stringstyle=\color{codepurple},
    basicstyle=\ttfamily\footnotesize,
    breakatwhitespace=false,         
    breaklines=true,                 
    captionpos=b,                    
    keepspaces=true,                 
    numbers=left,                    
    numbersep=5pt,                  
    showspaces=false,                
    showstringspaces=false,
    showtabs=false,                  
    tabsize=2
}
\lstset{style=mystyle}

% 标题信息
\title{神经网络与深度学习 - 项目2报告\\CIFAR-10分类与批归一化分析}
\author{22307110196 Oscar Yin}
\date{\today}

\begin{document}

\maketitle
\tableofcontents
\newpage

% 1. 引言
\section{引言}
% 项目背景和目标
\subsection{项目背景和目标}
本项目是“神经网络与深度学习”课程的第二个项目。项目的主要目标包括两个部分:

第一部分需要在CIFAR-10数据集上训练神经网络模型,以优化分类性能。CIFAR-10是一个广泛使用的视觉识别数据集,包含60,000张32×32彩色图像,分属10个不同类别,每个类别有6,000张图像。

在这部分任务中,我们需要构建包含全连接层、2D卷积层、2D池化层和激活函数等基本组件的网络,并尝试加入批归一化层、残差连接等高级组件。同时,我们还需要通过调整不同的神经元/滤波器数量、尝试不同的损失函数和激活函数,以及选择不同的优化器等策略来优化网络性能。

第二部分要求研究批归一化技术。批归一化是一种广泛采用的技术,它能够使深度神经网络的训练更快、更稳定。在这部分中,我们将使用与VGG-A相同架构的网络,比较有无批归一化版本模型的性能差异,并通过可视化损失景观等研究BN如何影响优化过程。

本报告将详细记录实验过程和主要发现,解释实验结果,并提供对批归一化作用机制的深入分析。

% 2. 在CIFAR-10上训练网络
\section{在CIFAR-10上训练网络}

% 网络架构设计
\subsection{网络架构设计}

在本项目中,我们采用了基于ResNet架构的卷积神经网络,并且针对 CIFAR-10 数据集做了适应性调整。我们的ResNet实现包含以下关键组件:

\begin{itemize}
    \item \textbf{输入层}:接收32×32×3的CIFAR-10图像
    \item \textbf{初始卷积层}:使用3×3卷积核(而非原始ResNet中的7×7),将输入通道从3扩展到64
    \item \textbf{残差块}:每个残差块包含两个3×3卷积层,每个卷积后接BatchNorm和激活函数,以及可选的1×1卷积用于调整通道数
    \item \textbf{残差连接}:将输入直接添加到残差块的输出,缓解梯度消失问题
    \item \textbf{全局平均池化}:在最后一个残差块后,使用全局平均池化减少特征图尺寸
    \item \textbf{全连接层}:将池化后的特征映射到10个类别
\end{itemize}

与原始ResNet相比,我们的实现针对CIFAR-10数据集进行了以下调整:
\begin{itemize}
    \item 使用\textbf{3x3}而非7×7的初始卷积核,适应32×32的小图像
    \item \textbf{不进行全局下采样池化},保留更多空间信息
    \item 允许灵活配置网络深度(通过num\_blocks参数)和激活函数类型
\end{itemize}


\subsection{实验设置与优化策略}

在实验探索阶段,为了平衡训练成本和实验效果,我们首先使用较小的模型和默认配置进行实验,大约8分钟可以完成一次训练。根据探索的结果,我们再训练较大的模型,并报告最优结果。表\ref{tab:default_settings}列出了实验的默认设置。

\begin{table}[htbp]
\centering
\caption{实验默认设置}
\label{tab:default_settings}
\begin{tabular}{ll}
\toprule
\textbf{参数} & \textbf{默认值} \\
\midrule
模型 & ResNet([1,1,1,1]),共10层 \\
训练轮次 & 25 \\
优化器 & SGD(动量0.9,权重衰减5e-4) \\
学习率调度器 & 余弦退火(周期与训练轮次相同) \\
损失函数 & 交叉熵 \\
激活函数 & ReLU \\
学习率 & 0.1 \\
批次大小 & 128 \\
数据增强 & 随机裁剪、水平翻转、归一化 \\
\bottomrule
\end{tabular}
\end{table}

部分默认设置的选择基于以下考虑:
\begin{itemize}
    \item \textbf{模型大小}:ResNet([1,1,1,1])是一个相对较小的模型,参数量适中(4.90M),适合快速实验
    \item \textbf{训练轮次}:25轮通常足以显示不同实验配置的区别
    \item \textbf{优化器}:SGD配合动量和权重衰减,有助于训练的稳定性
    \item \textbf{学习率调度}:余弦退火可以在训练初期使用较大学习率加速收敛,后期降低学习率提高精度
\end{itemize}

在完成初步探索后,我们根据实验结果调整了部分超参数,并尝试了不同的网络深度、激活函数和优化器,以寻找最佳性能配置。

% 优化策略
\subsection{实验结果与分析}

% 不同网络深度对比实验
\subsubsection{不同网络深度对比实验}

我们首先评估了不同深度(对应了不同的卷积核数量)的ResNet模型在CIFAR-10数据集上的性能。表\ref{tab:basic_model}展示了不同网络深度的实验结果。

\begin{table}[htbp]
\centering
\caption{不同网络深度的ResNet模型性能对比}
\label{tab:basic_model}
\begin{tabular}{lcccc}
\toprule
\textbf{模型配置} & \textbf{参数量} & \textbf{最佳验证准确率} & \textbf{最佳测试准确率} & \textbf{训练时间(分钟)} \\
\midrule
ResNet([1,1,1,1]) & 4.90M & 91.72\% & 92.07\% & 7.5 \\
ResNet([1,2,1,2]) & 9.92M & 91.82\% & 91.75\% & 9.0 \\
\bottomrule
\end{tabular}
\end{table}

从表\ref{tab:basic_model}可以看出,增加网络深度对模型性能的影响相对有限。更深的ResNet([1,2,1,2])模型在验证集上表现略微胜出,而在测试集上的表现略有下降。
一方面,这表明对于CIFAR-10数据集,较浅的ResNet模型已经足够捕捉数据中的特征,增加网络深度并不一定能带来性能提升;
另一方面,25轮的训练轮次可能足以让较浅的模型良好收敛,但对较深的模型则不足,这点可以从后面训练最佳模型的过程加以印证。

% 激活函数对比实验
\subsubsection{激活函数对比实验}

激活函数是神经网络中的关键组件,它决定了神经元的输出形式。我们比较了ReLU、Sigmoid和Tanh三种常用激活函数在ResNet模型上的表现。表\ref{tab:activation_functions}展示了实验结果。

\begin{table}[htbp]
\centering
\caption{不同激活函数的性能对比}
\label{tab:activation_functions}
\begin{tabular}{lcc}
\toprule
\textbf{激活函数} & \textbf{最佳验证准确率} & \textbf{最佳测试准确率} \\
\midrule
ReLU & 91.72\% & 92.07\% \\
Sigmoid & 40.40\% & 27.23\% \\
Tanh & 85.12\% & 84.76\% \\
\bottomrule
\end{tabular}
\end{table}

从表\ref{tab:activation_functions}可以看出,ReLU激活函数在验证集和测试集上都取得了最佳性能,分别达到91.72\%和92.07\%的准确率。Tanh激活函数表现次之,而Sigmoid激活函数的表现显著较差,测试准确率仅为27.23\%。

ReLU激活函数的优势在于其简单性和非饱和特性,可以有效缓解梯度消失问题,同时计算效率高。Tanh激活函数虽然理论上可以输出负值,但其饱和特性可能导致梯度消失问题,特别是在深层网络中,这可能是其性能不如ReLU的原因。Sigmoid激活函数的表现最差,这主要是因为其在深层网络中容易导致梯度消失问题,使得模型难以有效学习。
不同激活函数模型的验证损失可视化对比如图\ref{fig:activation}:
\begin{figure}[htbp]
    \centering
    \includegraphics[width=0.8\textwidth]{results/images/activation.png}
    \caption{不同激活函数模型的验证损失}
    \label{fig:activation}
\end{figure}

值得注意的是,Sigmoid激活函数在训练过程中出现了严重的性能退化,验证准确率从初始的28.50\%逐渐下降到最低点9.88\%,最终稳定在40.40\%,这表明Sigmoid激活函数在ResNet架构中确实存在严重的训练问题。

% 损失函数对比实验
\subsubsection{损失函数对比实验}

损失函数的选择对模型训练和性能有重要影响。我们比较了交叉熵损失和均方误差(MSE)损失在ResNet模型上的表现。表\ref{tab:loss_functions}展示了实验结果。

\begin{table}[htbp]
\centering
\caption{不同损失函数的性能对比}
\label{tab:loss_functions}
\begin{tabular}{lcc}
\toprule
\textbf{损失函数} & \textbf{最佳验证准确率} & \textbf{最佳测试准确率} \\
\midrule
交叉熵损失 & 91.72\% & 92.07\% \\
均方误差损失 & 91.38\% & 91.87\% \\
\bottomrule
\end{tabular}
\end{table}

从表\ref{tab:loss_functions}可以看出,均方误差损失则分别达到91.38\%和91.87\%的准确率,虽然不如交叉熵损失,但也能达到不错的性能。
均方误差作为损失函数的模型表现是出乎意料的,因为我们预计均方误差损失在分类任务中表现会非常差。

通常而言,均方损失的误差表面比交叉熵损失更平坦,在深层网络中容易导致梯度消失问题。本实验中两种损失函数都能达到较高的准确率,这表明ResNet架构本身具有较强的特征提取与缓解梯度消失的能力,即使使用次优的损失函数也能获得不错的性能。
然而,在追求最佳性能时,交叉熵损失仍然是分类任务的首选。
不同损失函数模型的验证集准确率可视化对比如图\ref{fig:loss_acc}:
\begin{figure}[htbp]
    \centering
    \includegraphics[width=0.8\textwidth]{results/images/loss_acc.png}
    \caption{不同损失函数模型的验证集准确率}
    \label{fig:loss_acc}
\end{figure}




% 权重衰减对比实验
\subsubsection{权重衰减对比实验}

给损失函数添加L2正则化项是防止模型过拟合的重要技术。由于在使用 SGD 优化器时,这个操作等价于权重衰减,
因此我们比较了不同权重衰减值对模型性能的影响。表\ref{tab:weight_decay}展示了实验结果。
我们比较了不同权重衰减值对模型性能的影响。表\ref{tab:weight_decay}展示了实验结果。

\begin{table}[htbp]
\centering
\caption{不同权重衰减值的性能对比}
\label{tab:weight_decay}
\begin{tabular}{lcc}
\toprule
\textbf{权重衰减} & \textbf{最佳验证准确率} & \textbf{最佳测试准确率} \\
\midrule
0.0005 & 91.72\% & 92.07\% \\
0.01 & 84.62\% & 85.34\% \\
0.0 & 90.42\% & 90.06\% \\
\bottomrule
\end{tabular}
\end{table}

从表\ref{tab:weight_decay}可以看出,权重衰减值对模型性能有显著影响。适中的权重衰减值(0.0005)取得了最佳性能,在测试集上达到92.07\%的准确率。无权重衰减(0.0)的表现次之,测试准确率为90.06\%。而较大的权重衰减值(0.01)则显著降低了模型性能,测试准确率降至85.34\%。

这表明对于CIFAR-10数据集和ResNet模型,适度的正则化(权重衰减值为0.0005)有助于提高模型性能,而过强的正则化(权重衰减值为0.01)则可能导致模型欠拟合。值得注意的是,即使不使用权重衰减,模型仍然能够达到90.06\%的测试准确率,
这表明ResNet架构本身具有一定的正则化效果,可能通过残差连接和批量归一化层实现。
不同权重衰减值模型的训练集准确率可视化对比如图\ref{fig:decay_train}:
\begin{figure}[htbp]
    \centering
    \includegraphics[width=0.8\textwidth]{results/images/decay_train.png}
\caption{不同权重衰减值模型的训练集准确率}
    \label{fig:decay_train}
\end{figure}

可以发现,权重衰减值为0.001的模型的训练集损失下降缓慢,这是因为权重衰减值过大,导致模型欠拟合。


% 优化器对比实验
\subsubsection{优化器对比实验}

优化器的选择对模型训练至关重要。我们比较了SGD、Adam和RMSprop三种常用优化器的性能。表\ref{tab:optimizers}展示了实验结果。

\begin{table}[htbp]
\centering
\caption{不同优化器的性能对比}
\label{tab:optimizers}
\begin{tabular}{lcc}
\toprule
\textbf{优化器} & \textbf{最佳验证准确率} & \textbf{最佳测试准确率} \\
\midrule
SGD & 91.72\% & 92.07\% \\
Adam & 50.70\% & 52.52\% \\
RMSprop & 50.92\% & 53.48\% \\
Adam(lr=0.001) & 91.42\% & 90.97\% \\
RMSprop(lr=0.0001) & 87.60\% & 86.72\% \\
\bottomrule
\end{tabular}
\end{table}

从表\ref{tab:optimizers}可以看出,SGD优化器在验证集和测试集上都取得了最佳性能,分别达到91.72\%和92.07\%的准确率。Adam和RMSprop的表现显著较差,Adam的测试准确率仅为52.52\%,而RMSprop的测试准确率仅为53.48\%。

Adam和RMSprop具有自适应学习率的优势,但在ResNet模型上表现不佳,这是因为我们使用了过高的学习率(0.1)。对于Adam和RMSprop这类自适应优化器,通常需要使用较小的学习率(如0.001或0.0001)才能获得良好性能,否则学习率在初始阶段可能会因为除法操作而变得非常大,导致模型训练不稳定。

事实也确实如此,尝试使用较小的学习率后,Adam和RMSprop的性能得到了显著提升。这表明自适应优化器虽然理论上具有优势,但在实际应用中需要更谨慎地调整学习率等超参数。

% 最佳模型配置
\subsubsection{最佳模型配置}

基于上述实验结果,我们确定了最佳模型配置,并在更大的模型上进行了训练。表\ref{tab:best_model}展示了最佳配置的性能。

\begin{table}[htbp]
\centering
\caption{最佳模型配置性能}
\label{tab:best_model}
\begin{tabular}{lcccc}
\toprule
\textbf{模型配置} & \textbf{参数量} & \textbf{最佳验证准确率} & \textbf{最佳测试准确率} & \textbf{训练时间(分钟)} \\
\midrule
ResNet([1,1,1,1]) & 4.90M & 91.72\% & 92.07\% & 7.5 \\
ResNet([1,2,1,2]) & 9.92M & 91.82\% & 91.75\% & 9.0 \\
\bottomrule
\end{tabular}
\end{table}

从表\ref{tab:best_model}可以看出,使用最佳配置(ReLU激活函数、权重衰减0.0005、SGD优化器、批量归一化)训练的ResNet([1,1,1,1])模型在测试集上达到了92.07\%的准确率,这是我们在这个实验阶段取得的最佳结果。

有趣的是,增加网络深度(ResNet([1,2,1,2]))并没有带来性能提升,反而略有下降。这表明对于CIFAR-10数据集,较浅的ResNet模型已经足够捕捉数据中的特征,增加网络深度并不一定能带来性能提升。这可能是因为CIFAR-10数据集的复杂度相对较低,不需要非常深的网络就能达到良好的分类效果。

值得注意的是,即使是最小的ResNet([1,1,1,1])模型,其参数量也达到了4.90M,这已经足够复杂以捕捉CIFAR-10数据集中的大部分特征。在后续实验中,我们可以尝试更深的模型(如ResNet([2,2,2,2])或ResNet([3,3,3,3])),以探索网络深度对性能的影响。

% 结果可视化

% 3. 批归一化研究
\section{批归一化研究}

% VGG-A对比实验
\subsection{VGG-A对比实验}
% 有无BatchNorm的VGG-A实现
\subsubsection{有无BatchNorm的VGG-A实现}

% 训练过程对比(收敛速度、准确率等)
\subsubsection{训练过程对比}

% 批归一化如何帮助优化
\subsection{批归一化如何帮助优化}
% 损失景观分析(Loss Landscape)
\subsubsection{损失景观分析}

% 梯度可预测性分析
\subsubsection{梯度可预测性分析}

% 不同学习率下的表现对比
\subsubsection{不同学习率下的表现对比}

% 结果讨论
\subsection{结果讨论}
% 批归一化的作用机制解释
\subsubsection{批归一化的作用机制解释}

% 实验结果与理论分析的对应关系
\subsubsection{实验结果与理论分析的对应关系}

% 4. 结论
\section{结论}
% 主要发现总结
\subsection{主要发现总结}

% 实验启示
\subsection{实验启示}

% 未来工作方向
\subsection{未来工作方向}

% 参考文献
\section{参考文献}
\begin{thebibliography}{9}
\bibitem{PyTorch} PyTorch tutorial, \url{https://pytorch.org/tutorials/beginner/blitz/cifar10_tutorial.html}
\bibitem{CIFAR} CIFAR-10 dataset, \url{https://www.cs.toronto.edu/~kriz/cifar.html}
\bibitem{BN} Ioffe, Sergey, and Christian Szegedy. "Batch normalization: Accelerating deep network training by reducing internal covariate shift." ICML 2015.
\bibitem{BN_smoothing} Santurkar, Shibani, et al. "How does batch normalization help optimization?" NeurIPS 2018.
\end{thebibliography}

% 附录
\section{附录}
% 代码链接
\subsection{代码链接}
代码仓库链接:\url{https://github.com/...}

% 模型权重链接
\subsection{模型权重链接}
模型权重下载链接:\url{https://...}

% 其他补充材料
\subsection{其他补充材料}

\end{document} 